\section{介绍}
\zihao{-4}
\label{section:introduction}

% 对话机器人作为现在和未来各种服务的入口,是当前人工智能领域的热门研究方向。

% 国内外已出现了诸多类型的对话机器人,
% 以应用目的(以“\textbullet”标识)和技术手段(以“\dag”标识)进行划分,
% 可有以下两种分类方法。

% \begin{enumerate}[leftmargin=3.5em,itemindent=0em,label=(\arabic*),itemsep=0pt,topsep=0pt]
%     \item[\textbullet] 
%     目标驱动(Goal-based):目标驱动对话机器人具有明确的服务目标、对象或用途,
%     如Facebook Negotiation谈判机器人、阿里Alime客服机器人、Disrupt 911bot报警机器人等,
%     是面向特定用途的对话机器人;
%     \item[\textbullet]
%     无目标驱动(Non Goal-based):
%     与目标驱动机器人不同,
%     以微软Cortana、苹果Siri、谷歌Assitant为代表的无目标驱动对话机器人指代
%     并非面向特定领域或服务目的,提供搜索资料、进行终端设置、调用终端功能和聊天等各种服务的对话机器人。
%     这种对话机器人没有明确任务目标,也可被称为开放领域的对话机器人。
%     \item[\dag]
%     检索式:检索式对话机器人指事先具有知识库,对话系统接受到用户的输入句子后,
%     通过在知识库中以搜索匹配的方式进行应答内容提取;
%     \item[\dag]
%     提出具体研究和实施方案,
%     即如何用机器学习等人工智能的技术、方法、模型和算法来实现面向特定领域的对话机器人
%     (如替代电信114接线员)。
% \end{enumerate}

% 目标驱动(Goal-based):目标驱动对话机器人具有明确的服务目标、对象或用途,
% 如Facebook Negotiation谈判机器人、阿里Alime客服机器人、Disrupt 911bot报警机器人等,
% 是面向特定用途的对话机器人;

% 无目标驱动(Non Goal-based):
% 与目标驱动机器人不同,
% 以微软Cortana、苹果Siri、谷歌Assitant为代表的无目标驱动对话机器人指代
% 并非面向特定领域或服务目的,提供搜索资料、进行终端设置、调用终端功能和聊天等各种服务的对话机器人。
% 这种对话机器人没有明确任务目标,也可被称为开放领域的对话机器人。

从早起的图书情报检索系统、电信114服务系统到现在的在线呼叫中心,
快速并准确的获取与反馈信息一直是用户和企业的追求目标。
% 尤其是在信息呈现规模巨大、模态多样、关联复杂和真假难辨的大数据时代。
% 一方面,用户的需求愈加多样,基于关键词抽取的主题相关模型或或基于浅层语义分析的检索系统已不能满足用户的需求;
% 另一方便,企业的服务更加精细,希望提供更加类人的自然语言应答,以减轻人工客服的压力,优化企业的服务资源配置。
伴随而生的智能问答系统已有近70年的发展历史。
早期的智能问答系统大多数针对特定问题设计,
且由于技术和环境的限制数据量十分有限,不易进行扩展和训练,如\citet{Green-2}和\citet{Woods-3}。
这些诞生在上世界六七十年代的智能问答系统仅接受特定形式的自然语言语句,且供系统训练的数据也很少,
无法进行较广范围的问答从而未被广泛使用。

进入九十年代后,借助互联网技术的发展,大量可供训练的问答对在网上可被搜索和爬取,进而构建特定主题下的语料库
\citep{dang2007overview}。
这些语料库的出现极大促进了智能问答系统的发展,研究人员在这些语料库上训练和测试各种问答模型,
先后提出了基于逻辑推理\citep{moldovan2001logic}、
基于模式匹配\citep{soubbotin2001patterns}和
基于机器学习\citep{yang2002integration}等许多方法。
在此阶段,人们主要利用信息检索或浅层语义理解去候选应答集中寻找应答从而构建智能问答系统,
故将此类系统归纳为检索式问答系统。但检索式问答存在固有缺陷。应答的准确与否很大程度上取决于当前问句的信息充分程度,
一方面,不能很好的关联该问句的上下;另一方面,问答系统的训练依赖当前问句抽取后表述的标签或是关联规则。
因此检索式问答系统对短问句的应答效果欠佳,且规则构建仍需要较高的人力或专家知识。

随着深度学习方法在学界和业界均取得了较好的效果,研究人员将端到端的思想应用在了智能问答系统,面向不同特定领域提出了
基于端到端的统计机器翻译\citep{cho2014learning}、
机器阅读理解\citep{tan2017s}和机器谈判代理\citep{lewis2017s}等许多模型。
在此阶段,人们利用词嵌入(Word Embedding)、编-解码(Encode-Decode)和
改进型循环神经网络(如LSTM、GRUs)等方法生成应答文本,
故将此类系统归纳为生成式问答系统。
虽然生成式系统能够一定程度地解决长问句和问句的上下文理解难题,
但生成式问答同样存在弊端。
一方面,相较于基于规则或搜索的检索式系统,在处理已存在语义库和知识库的问答对时,
可能需要上千次对话训练才能达到检索式问答系统几次简单设置(构建规则)的效果;
另一方面,虽然生成式系统能够记忆上文甚至推演下文\citep{lewis2017s}给用户一种在和人类对话的感觉,
然而,这种模型很难训练,在进行长应答时很可能会犯语法错误。
因此生成式系统相对检索式系统需要更多的对话训练数据。

目前,基于检索式或生成式的问答系统都可应用于对话机器人,但在面对开放领域和特定领域时,两者各有优劣。
在面对开放领域时,典型如较纯粹聊天场景,
此时用户的话题并不面向特定领域和具有特定目的或任务,希望得到类人的自然语言问答体验,
因此生成式问答模型在面向开放领域即无任务驱动下具有天然的优势,
此类对话机器人有微软“小冰”、Facebook“Messenger”和Github“Hubot”等。
而在面对特定领域,典型如呼叫中心,用户与对话机器人均面向特定领域和具有特定目的或任务,
此时用户更希望对话机器人能够准确回答领域内问题。
因此检索式问答模型在面向特定领域即任务驱动下更具有优势,
此类对话机器人如面向报警的Disrupt“911bot”\citep{911bot}、
面向在线客服的阿里“ALIME”\citep{alimi}、京东“JIMI”\citep{jimi}和网易“七鱼”\citep{wangyiqiyu}等。

近年来,
基于WordNet、HowNet等词汇知识库和Wikipedia与电商数据这种动态更新的知识资源库,
大规模知识图谱日益成熟;
同时,基于统计机器学习的自然语言处理和基于深度学习的知识推理技术有了快速发展
;此外,CUDA加速计算的出现大大降低了机器学习与深度学习带来的庞大计算开销。
这三方面的进步分别为智能问答系统的发展奠定了资源、技术和成本基础,
给智能问答系统的发展带来了新的契机。

值得关注的是,随着深度学习的浪潮,使用深度学习完成任务驱动下的问答模型成为具有技术优势的学界主流;
而在具有资源优势的业界,以智能客服为代表的任务驱动问答模型却几乎都采用更实用的检索式问答模型。
如何通过深度学习对接包含词典、规则和知识图谱在内的知识,使检索式问答模型与生成式问答模型巧妙融合,
达到学界和业界优势互补,是当前厄待解决的问题。

本文基于在线呼叫中心面向客服领域,首先对比国内外具有代表性的呼叫中心智能客服系统,
合理推测其工作原理和实现方式,并进行对比与评价;
随后融合统计机器学习与知识,总结了一套“生成-检索-生成”流程,
提出一种混合式的端到端智能问答系统。
